\documentclass[a4paper]{article}
\usepackage{tcolorbox}
\usepackage{listings}
\tcbuselibrary{skins}
\usepackage[utf8]{inputenc}
\usepackage[T1]{fontenc}

\title{
\vspace{-3em}
\begin{tcolorbox}
\Huge\sffamily CSS Display Property
\end{tcolorbox}
\vspace{-3em}
}

\date{\today}

\usepackage{background}
\SetBgScale{1}
\SetBgAngle{0}
\SetBgColor{red}
\SetBgContents{\rule[0em]{4pt}{\textheight}}
\SetBgHshift{-2.3cm}
\SetBgVshift{0cm}
\usepackage[margin=2cm]{geometry}

\makeatletter
\newcommand{\cornell}{\@ifnextchar[{\@with}{\@without}}
\newcommand{\@with}[4]{
\begin{tcolorbox}[enhanced,colback=gray,colframe=black,fonttitle=\large\bfseries\sffamily,sidebyside=true, nobeforeafter,colupper=blue,sidebyside align=top, lefthand width=.3\textwidth,
opacityframe=0,opacityback=.3,opacitybacktitle=1, opacitytext=1,
segmentation style={black!55,solid,opacity=0,line width=3pt},
title=#2
]
\tcblower
\sffamily
\begin{tcolorbox}[colback=blue!05,colframe=blue!10,width=\textwidth,nobeforeafter]
#3
\end{tcolorbox}
\end{tcolorbox}
}
\newcommand{\@without}[3]{
\begin{tcolorbox}[enhanced,colback=white!15,colframe=white,fonttitle=\bfseries,sidebyside=true, nobeforeafter,colupper=blue,sidebyside align=top, lefthand width=.3\textwidth,
opacityframe=0,opacityback=0,opacitybacktitle=0, opacitytext=1,
segmentation style={black!55,solid,opacity=0,line width=3pt}
]

\begin{tcolorbox}[colback=red!05,colframe=red!25,sidebyside align=top,
width=\textwidth,nobeforeafter]#1\end{tcolorbox}%
\tcblower
\sffamily
\begin{tcolorbox}[colback=blue!05,colframe=blue!10,width=\textwidth,nobeforeafter]
#2\\
#3
\end{tcolorbox}
\end{tcolorbox}
}
\makeatother

\parindent=0pt

\begin{document}
\maketitle
\SetBgContents{\rule[0em]{4pt}{\textheight}}

\cornell{Display property}{block; inline; inline-block; none}

\cornell[{Block}{takes up the whole width}]

\cornell{span element}{to select one part of an element. they are like divs; so without css they make no difference.span is an inline display element.}

\cornell[{Block}{p - div - h1 to h6 - table - lists}]

\cornell[{inline}{span - image - anchor}]   

\cornell[{TIP}{inline elements can't have their width set up}]

\cornell{inline-block}{can be set on width, and be inline at the same time}

\cornell{none}{the element will be completely removed}

\cornell[{TIP}{visibility: hidden; will make them disappear. but they will be taking their space anyway}]

\end{document}
