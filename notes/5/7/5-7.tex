\documentclass[a4paper]{article}
\usepackage{tcolorbox}
\usepackage{listings}
\tcbuselibrary{skins}
\usepackage[utf8]{inputenc}
\usepackage[T1]{fontenc}

\title{
\vspace{-3em}
\begin{tcolorbox}
\Huge\sffamily CSS Static and Relative Positioning
\end{tcolorbox}
\vspace{-3em}
}

\date{\today}

\usepackage{background}
\SetBgScale{1}
\SetBgAngle{0}
\SetBgColor{red}
\SetBgContents{\rule[0em]{4pt}{\textheight}}
\SetBgHshift{-2.3cm}
\SetBgVshift{0cm}
\usepackage[margin=2cm]{geometry}

\makeatletter
\newcommand{\cornell}{\@ifnextchar[{\@with}{\@without}}
\newcommand{\@with}[4]{
\begin{tcolorbox}[enhanced,colback=gray,colframe=black,fonttitle=\large\bfseries\sffamily,sidebyside=true, nobeforeafter,colupper=blue,sidebyside align=top, lefthand width=.3\textwidth,
opacityframe=0,opacityback=.3,opacitybacktitle=1, opacitytext=1,
segmentation style={black!55,solid,opacity=0,line width=3pt},
title=#2
]
\tcblower
\sffamily
\begin{tcolorbox}[colback=blue!05,colframe=blue!10,width=\textwidth,nobeforeafter]
#3
\end{tcolorbox}
\end{tcolorbox}
}
\newcommand{\@without}[3]{
\begin{tcolorbox}[enhanced,colback=white!15,colframe=white,fonttitle=\bfseries,sidebyside=true, nobeforeafter,colupper=blue,sidebyside align=top, lefthand width=.3\textwidth,
opacityframe=0,opacityback=0,opacitybacktitle=0, opacitytext=1,
segmentation style={black!55,solid,opacity=0,line width=3pt}
]

\begin{tcolorbox}[colback=red!05,colframe=red!25,sidebyside align=top,
width=\textwidth,nobeforeafter]#1\end{tcolorbox}%
\tcblower
\sffamily
\begin{tcolorbox}[colback=blue!05,colframe=blue!10,width=\textwidth,nobeforeafter]
#2\\
#3
\end{tcolorbox}
\end{tcolorbox}
}
\makeatother

\parindent=0pt

\begin{document}
\maketitle
\SetBgContents{\rule[0em]{4pt}{\textheight}}

\cornell[{TIP}{1.content is everything\\order comes from code\\children sit on top of parents}]

\cornell{Position}{static/relative/absolute/fixed}

\cornell{static}{all html elements are static by default. keep to the code flow}

\cornell{relative}{positions the element relative to it's static position.\\top/bottom/left/right}

\cornell[{TIP}{in relative positioning, its like the element still has its own space. and when it moves around, it doesnt push other objects away. it comes on top of them. but still it has its own space in its relative place}]

\cornell[{TIP}{relative positioning doesnt do anything by itself. its only usable when its used with one of the directions said above}]

\end{document}
