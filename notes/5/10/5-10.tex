\documentclass[a4paper]{article}
\usepackage{tcolorbox}
\usepackage{listings}
\tcbuselibrary{skins}
\usepackage[utf8]{inputenc}
\usepackage[T1]{fontenc}

\title{
\vspace{-3em}
\begin{tcolorbox}
\Huge\sffamily Font Styling in out Personal Site
\end{tcolorbox}
\vspace{-3em}
}

\date{\today}

\usepackage{background}
\SetBgScale{1}
\SetBgAngle{0}
\SetBgColor{red}
\SetBgContents{\rule[0em]{4pt}{\textheight}}
\SetBgHshift{-2.3cm}
\SetBgVshift{0cm}
\usepackage[margin=2cm]{geometry}

\makeatletter
\newcommand{\cornell}{\@ifnextchar[{\@with}{\@without}}
\newcommand{\@with}[4]{
\begin{tcolorbox}[enhanced,colback=gray,colframe=black,fonttitle=\large\bfseries\sffamily,sidebyside=true, nobeforeafter,colupper=blue,sidebyside align=top, lefthand width=.3\textwidth,
opacityframe=0,opacityback=.3,opacitybacktitle=1, opacitytext=1,
segmentation style={black!55,solid,opacity=0,line width=3pt},
title=#2
]
\tcblower
\sffamily
\begin{tcolorbox}[colback=blue!05,colframe=blue!10,width=\textwidth,nobeforeafter]
#3
\end{tcolorbox}
\end{tcolorbox}
}
\newcommand{\@without}[3]{
\begin{tcolorbox}[enhanced,colback=white!15,colframe=white,fonttitle=\bfseries,sidebyside=true, nobeforeafter,colupper=blue,sidebyside align=top, lefthand width=.3\textwidth,
opacityframe=0,opacityback=0,opacitybacktitle=0, opacitytext=1,
segmentation style={black!55,solid,opacity=0,line width=3pt}
]

\begin{tcolorbox}[colback=red!05,colframe=red!25,sidebyside align=top,
width=\textwidth,nobeforeafter]#1\end{tcolorbox}%
\tcblower
\sffamily
\begin{tcolorbox}[colback=blue!05,colframe=blue!10,width=\textwidth,nobeforeafter]
#2\\
#3
\end{tcolorbox}
\end{tcolorbox}
}
\makeatother

\parindent=0pt

\begin{document}
\maketitle
\SetBgContents{\rule[0em]{4pt}{\textheight}}

\cornell[{TIP}{if the parent of an absolute element isnt relative, then the element gets its layout relative to the body}]

\cornell[{TIP}{changing the font style and size, will affect the page layout}]

\cornell{fonts}{sans-serif+serif:for typical usage / monospace: for coding / cursive + fantasy: never use!!! not easy to read}

\cornell[{TIP}{default font is serif}]

\cornell[{TIP}{font-family: font1, font2. the priority is with font1, but if the browser didnt have font1, it will use font2}]

\cornell{Safe fonts}{fonts that most browsers support}

\cornell[{TIP}{no font is 100\% safe}]

\cornell[{SITE}{cssfontstack.com\\fonts.google.com}]

\cornell[{TIP}{to make the safe percentage 99, we should embed our language using google font site}]

\end{document}
