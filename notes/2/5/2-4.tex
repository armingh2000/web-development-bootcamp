\documentclass[a4paper]{article}
\usepackage{tcolorbox}
\usepackage{listings}
\tcbuselibrary{skins}
\usepackage[utf8]{inputenc}
\usepackage[T1]{fontenc}

\title{
\vspace{-3em}
\begin{tcolorbox}
\Huge\sffamily What is The HTML Boilerplate
\end{tcolorbox}
\vspace{-3em}
}

\date{\today}

\usepackage{background}
\SetBgScale{1}
\SetBgAngle{0}
\SetBgColor{red}
\SetBgContents{\rule[0em]{4pt}{\textheight}}
\SetBgHshift{-2.3cm}
\SetBgVshift{0cm}
\usepackage[margin=2cm]{geometry}

\makeatletter
\newcommand{\cornell}{\@ifnextchar[{\@with}{\@without}}
\newcommand{\@with}[4]{
\begin{tcolorbox}[enhanced,colback=gray,colframe=black,fonttitle=\large\bfseries\sffamily,sidebyside=true, nobeforeafter,colupper=blue,sidebyside align=top, lefthand width=.3\textwidth,
opacityframe=0,opacityback=.3,opacitybacktitle=1, opacitytext=1,
segmentation style={black!55,solid,opacity=0,line width=3pt},
title=#2
]
\tcblower
\sffamily
\begin{tcolorbox}[colback=blue!05,colframe=blue!10,width=\textwidth,nobeforeafter]
#3
\end{tcolorbox}
\end{tcolorbox}
}
\newcommand{\@without}[3]{
\begin{tcolorbox}[enhanced,colback=white!15,colframe=white,fonttitle=\bfseries,sidebyside=true, nobeforeafter,colupper=blue,sidebyside align=top, lefthand width=.3\textwidth,
opacityframe=0,opacityback=0,opacitybacktitle=0, opacitytext=1,
segmentation style={black!55,solid,opacity=0,line width=3pt}
]

\begin{tcolorbox}[colback=red!05,colframe=red!25,sidebyside align=top,
width=\textwidth,nobeforeafter]#1\end{tcolorbox}%
\tcblower
\sffamily
\begin{tcolorbox}[colback=blue!05,colframe=blue!10,width=\textwidth,nobeforeafter]
#2\\
#3
\end{tcolorbox}
\end{tcolorbox}
}
\makeatother

\parindent=0pt

\begin{document}
\maketitle
\SetBgContents{\rule[0em]{4pt}{\textheight}}

\newsavebox{\htmlc}
\begin{lrbox}{\htmlc}
  \begin{lstlisting}
    <!DOCTYPE html>
    <html>
      <head>
        <meta charset="utf-8">
        <title></title>
      </head>
      <body>

      </body>
    </html>
  \end{lstlisting}
\end{lrbox}

\cornell{Italisize tags}{i tag and em tag}

\cornell[{i vs em}{These two are exactly the same in appearance. But the difference is that em tag, actually means emphasizing on the content. While i tag is just about the style. In html, we should try to structure our web site rather than just decorizing it. So it is usually a good idea to use em instead of i.}]

\cornell{Bold tags}{b tag and strong tag}

\cornell[{TIP}{Again like italisize tags, the same rule applies to bold tags; and it's best practice to use strong tag that shows strong importance instead of b tag}]



\end{document}
